\documentclass{article}
\usepackage{hyperref, listings, color}

% shortcut for scotv's github homepage
\newcommand{\scotv}{https://github.com/scotv}

% shortcut for inline code snippet, like `code` in markdown
% \newcommand{\cd}[1]{\colorbox[rgb]{0.86,0.86,0.86}{\lstinline$#1$}}
\newcommand{\cd}[1]{\lstinline$#1$}

% shortcut for section, subsection, subsubsection
% mb stands for member
\newcommand{\mb}[1]{\subsection*{#1}}
\newcommand{\mmb}[1]{\subsubsection*{#1}}

% set style for multiple lines code snippet
\lstset{numbers=left, numberstyle=\tiny
	, stepnumber=2, numbersep=5pt
	, backgroundcolor=\color[rgb]{0.86,0.86,0.86}
	, basicstyle=\footnotesize\ttfamily
	, breaklines=true}

\title{The API Documents for algo-x}
\author{Scott, Liu}
\date{August, 12, 2014}

\begin{document}
\maketitle

This book has just been migrated from previous wiki page of \href{\scotv/algo-js}{algo-js}. There are lots of formats need to be rebuilt. And, lots of namespaces are still in JaveScript, that I am converting right now.

The purpose of this book is listing the API of projects algo-x:

\begin{itemize}
	\item \href{\scotv/algo-js}{algo-js}: Implementation of some algorithms using JavaScript
	\item \href{\scotv/algo-scala}{algo-scala}: Implementation of some algorithms using Scala
	\item \ldots{} 
\end{itemize}

\section{Type Extension}
All types of structures are defined in \cd{window.T}.

We pass some funtion as parameter sometimes, such as \cd{.map(fn), .forEach(fn), .sort(compare)} \ldots 
The default \cd{fn} for \cd{.map(fn)} is \cd{x => x}, while the default for comparasion is \cd{(x, y) => x - y}.

\subsection{Type Constructors}
\mmb{new T.LinkedList()}
Constructs a new linked list.
\mmb{new T.Stack()}
Constructs a new stack.
\mmb{new T.Queue()}
Constructs a new queue.
\mmb{new T.MaxHeap(): heap}
Gets a new max heap.
\mmb{new T.MinHeap(compare): heap}
Gets a new min heap.
\mmb{new T.QuickFind(n: number): unionfind}
Gets a union find by QuickFind algorithm.
\mmb{new T.WeightedQuickUnion(n: number): unionfind}
Gets a union find by Weighted QuickFind algorithm.
\mmb{new T.BinarySearchTree(compare)}
Gets a Binary Search Tree with the specific comparasion rule.
\mmb{new T.Graph(n: number, directed: bool = false)}
Gets a unweighted graph with \cd{n} vertex, which is undirected by default.
\mmb{new T.GraphW(n: number, directed: bool = false)}
Gets a weighted graph with \cd{n} vertex, which is undirected by default.

\subsection{General Members of Linear Collection}
Linear collection like \cd{LinkedList, Stack, Queue} \ldots have the default traversal order. 

\cd{LinkedList} traverses each element in \cd{0 ... n-1} order;

\cd{Stack} traverses each element in \cd{LIFO} order;

\cd{Queue} traverses each element in \cd{FIFO} order;

There are some general members defined in linear collection. 

\mmb{size(): number}
Gets the size (length) of the linear collecion.
\mmb{isEmpty(): boolean}
Returns \cd{true} when the linear collection is empty, otherwise \cd{false}.
\mmb{forEach(x =\textgreater void): void}
Applies a function to each element in default order.
\mmb{map(x =\textgreater any): [any]}
Gets a new array mapped from \cd{x => any} in default order.
\mmb{toArray(): [ ]}
Gets a new array containing each element of this linear collection in default order.

\subsection{LinkedList}
\mmb{insert(any, i): void}
Inserts an element after ith element, indexing from 0, if \cd{i >= size()}, insert elem at the end of list.
\mmb{push(any): void}
Inserts an element at the end of list.
\mmb{reverse(): void}
Reverses the list itself.
\mmb{remove(any): void}
Removes the first node with such element, if no such node, keeps array unchanged.

\subsection{Stack}
\mmb{push(any): void}
Pushes \cd{any} item into \cd{stack}.
\mmb{peek(): any}
Gets the last added item from \cd{stack}, throws \cd{error} when empty.
\mmb{pop(): any}
Gets and removes the last added item from \cd{stack}, throws \cd{error} when empty.

\subsection{Queue}
\mmb{enqueue(any): void}
Addes item into \cd{queue}.
\mmb{peek(): any}
Gets the first added item from \cd{queue}, throws \cd{error} when empty.
\mmb{dequeue(): any}
Gets and removes the first added item from \cd{queue}, throws \cd{error} when empty.

\subsection{Heap}
Usage of \cd{T.MinHeap} is same as \cd{T.MaxHeap}, except invoking the constructor.
Notice that there is a pseudo element at \cd{heap[0]} which we might not use.

The reason we do not implement public \cd{forEach}, \cd{map} or \cd{toArray} is
that any of these functions will destroy the \cd{heap} itself.

\mmb{isEmpty(): boolean}
Returns \cd{true} if there is no real element in \cd{heap}, otherwise \cd{false}.
\mmb{size(): number}
Gets the number of elements in \cd{heap}.
\mmb{push(any): number}
Inserts an element into the \cd{heap}, and gets \cd{size()} of the new \cd{heap}.
\mmb{pop(): any}
Get the max element from \cd{MaxHeap}, or the min element from \cd{MinHeap}, 
the remove it, keeping \cd{heap} sorted.
\mmb{minheap.update(key, when, how): boolean}
\cd{minheap.update(key: x => boolean, when: x => boolean, how: x => void): boolean}

Finds the elem holding \cd{key(x)}, if \cd{when(x)}, then updates that elem by \cd{how(x)}, returns \cd{false} iff no elem hoding \cd{key(x)}.

\subsection{QuickFind}
When we construct a \cd{QuickFind}, we pass \cd{n} as the capacity parameter of the quick find. We mark the elements in quick find from `0` to `n-1` (inclusive).

Different algorithms of quick find bring us different growth of the cost, see below:

\begin{center}
\begin{tabular}{l | c | c}
\hline
Type & \cd{connected(p, q)} & \cd{union(p, q)} \\
\hline
\cd{T.QuickFind(N)} & $O(1)$ & $O(N)$ \\
\cd{T.WeightedQuickFind(N)} & $O(\ln N)$ & $O(\ln N)$ \\
\hline
\end{tabular}
\end{center}

\mmb{connected(p, q): boolean}
Returns \cd{true} iff \cd{p} is connected with \cd{q}, otherwise \cd{false}.
\mmb{count(): number}
Gets the number of components which are not connected with each other.
\mmb{union(p, q): number}
Unions \cd{p} with \cd{q}, and returns the \cd{count()} after union.


\subsection{BinarySearchTree}
We can indicate the comparasion rule by constructor parameter which can be used for greater left instead of smaller left by default.

For \cd{search}, \cd{insert} and \cd{delete} operations which travel the tree, there are two versions, the one is iterative operation, the other is recursive one whose function name is initialized with a \cd{r}. 

Always choose iterative operations as a recommandation in \cd{JavaScript}.

And, here is definition for some properties of tree, according to Wikipedia\footnote{Terminology of Tree on Wikipedia, \url{https://en.wikipedia.org/wiki/Tree_(data_structure)\#Terminology}}:

\begin{itemize}
  \item The height of a node is the length of the longest downward path to a leaf from that node. 
  \item The depth of a node is the length of the path to its root (i.e., its root path). The root node has depth zero, leaf nodes have height zero.
  \item The height of the root is the height of the tree. An empty tree (tree with no nodes, if such are allowed) has depth and height $-1$.
\end{itemize}

We have defined different traversal way in \cd{T.TRAVERSAL.*}. The default traversal way is \cd{T.TRAVERSAL.IN_ORDER}.

\mmb{search(elem): node}
Searches \cd{elem} in this tree, returns node which contains\cd{elem}, or \cd{null} if not exsits.

Recursive version is \cd{rSearch(elem): node}.

\mmb{insert(elem): void}
Inserts \cd{elem} under \cd{BST} order, no duplication.

Recursive version is \cd{rInsert(elem): void}.

\mmb{forEach(t: TRAVERSAL, fn): void}
Travels and applies function \cd{fn} on this \cd{BST} tree with specific \cd{T.TRAVERSA} order.

Recursive version is \cd{rForEach(t: TRAVERSAL, fn): void}.

\mmb{map(t: TRAVERSAL, fn): [any]}
Maps each element of this \cd{BST} tree into an array with specific \cd{T.TRAVERSA} order.

Recursive version is \cd{rMap(t: TRAVERSAL, fn): void}.

\subsection{Graph}
Representation of unweighted graph, and it is undirected graph by default.

To build a graph, we must pass \cd{n} as number of vertex. \cd{n}, the number of vertex, is readonly after builted graph.
However, when we call \cd{v()} or \cd{e()} as below, the number we get is the valid number of vertex or edges, which has not been visited or marked.

\mmb{n: number}
Gets the actual number of vertex, whether it has been visited, marked or not.
\mmb{v(): number}
Gets the number of vertex, which has not been visited.
\mmb{e(): number}
Gets the number of edges, which has not been marked invalid, and the sourcing vertex has not been visited.
\mmb{clone(): Graph}
Gets a new cloned graph from this original graph itself.
\mmb{toString(verbose: boolean = false): string}
Gets the information of this graph, containing only number of vertex and edges by default.

\subsection{GraphW}
The members in weighted graph are same as the graph, except some differences between the adjancency list representation:

\begin{center}
\begin{tabular}{l | l}
\hline
Type & Element in Adjancency Lis \\
\hline
\cd{T.Graph} & \cd{[v, [u1, u2, u3, ...]]} \\
\cd{T.GraphW} & \cd{[v, [(u1, w1), (u2, w2), (u3, w3), ...]]} \\
\hline
\end{tabular}
\end{center}
\section{Array}
The basic extensions for Array.

\subsection*{Array.zip(arr1, arr2): [ ]} 
Returns an new array, each item in which is \cd{[x, y]}, where \cd{x} from \cd{arr1} and \cd{y} from \cd{arr2}.

\begin{lstlisting}
Array.zip([1, 2, 3], ['A', 'B', 'C'])
// [[1, 'A'], [2, 'B'], [3, 'C']]
\end{lstlisting}

\subsection*{Array.swap(arr, i, j): void} 
Swaps the elements at index \cd{i} and \cd{j}, and does modify the original \cd{arr}.

\subsection*{.clone(): [ ]}
Get a new array cloned from the \cd{arr} itself.

\subsection*{.zip(that): [ ]}
Gets \cd{Array.zip(this, that)}.

\section{Math Extension}
The basic extensions for Math.

\mb{Math.mod(i, n): number}
Returns a positive number $x$, where $i = k * n + x, (x > 0)$.
\mb{Math.range(start = 0, end, step = 1): [ ]}
Generates an array of range from \cd{start} (inclusive) to \cd{end} (exclusive), with that \cd{step}.
\mb{Math.randomInteger(a = 0, b): number}
Gets a random integer $x$, where $ a \leq x \leq b $.
 
\subsection{Statistics}
The basic statistics extension that is located in \cd{Math.Stats.*}

\mmb{equals(x: number, y: number): boolean}
Return true iif \cd{Math.abs(x - y) < 1e-29}.
\mmb{equals(x: Array, y: Array): boolean}
Return true iif the length of array x and y is same and each items in same index are \cd{equals()}.
\mmb{max(arr): number}
Gets the max number of an array.
\mmb{min(arr): number}
Gets the min number of an array.
\mmb{sum(arr): number}
Gets the sum of an array.
\mmb{normalize(arr): []}
Normalizes the array, for instance: \cd{normalize([1, 7, 3])} should equal
$[\frac{1}{1+7+3}, \frac{7}{1+7+3}, \frac{3}{1+7+3}]$.
\mmb{mean(arr): number}
Gets the average of an array.
\mmb{var(arr): number}
Gets sample variance of an array.
\mmb{stddev(arr): number}
Gets the standard deviation of an array.
\mmb{linearLeastSquare(X,  Y, fn=(x=\textgreater x) ): [number, number]}
Gets the \cd{[a, b]}, where $\mathrm{fn}(\vec{Y}) = a \cdot \mathrm{fn}(\vec{X}) + b$. Passes \cd{Math.log} to get linear least square for $Y = c \cdot X^a$, where $c = e^b$.


\subsection{Math.Point}
The point for the vector.

\mmb{new Math.Point(arr)}
Builds new point by coordinates array.

\subsection{Math.Vector}
The vector in $n$ dimension.

\mmb{new Math.Vector(arr)}
Builds new vector by coordinates array.
\mmb{new Math.Vector(arr1, arr2)}
Builds new vector by coordinates array of two point.
\mmb{new Math.Vector(point1, point2)}
Builds new vector by two Math.Point.

\mmb{Math.Vector.norm(vec): number}
Gets the length of the vec.
\mmb{Math.Vector.dot(v1, v2): number}
Calculates the \cd{dot} result of two vectors.
\mmb{Math.Vector.cos(v1, v2): number}
Calculates the \cd{cos} result of two vectors.

\mmb{norm(): number}
Gets the length of the vec itself.
\mmb{dot(that): number}
Calculates the \cd{dot} result of this and that.
\mmb{cos(that): number}
Calculates the \cd{cos} result of this and that.
\section{Linear Algorithm}
We put some algortithms on linear collection into \cd{List.*}. To use linear collection type, please go to \cd{T.Stack}, \cd{T.LinkdedList} \ldots{}

A overview of time complexity laies below:


% \begin{center}
\begin{tabular}{l | c}
\hline
Call & Time Complexity \\
\hline
\cd{validPopStackSeries} & $\mathrm{O}(n)$ \\
\cd{medianMaintenence} & $\mathrm{O}(n \ln n)$ \\
\hline
\end{tabular}
% \end{center}

\mb{validPopStackSeries(push: [ ], pop: [ ]): bool}
Returns true iff we can get the \cd{pop} series from the \cd{push} series, using a stack.
\mb{medianMaintenence(arr): [ ]}
For each time $t$ we visit \cd{arr} from $0$ to $n-1$, we maintenence the median of all $[0, t]$ elements from input, and push this median into output array.

\section{Sorting Algorithm}
In our sorting, we pass a function as parameters, named \cd{compare(x, y) : number}. We sort the array by comparing each two value with this \cd{compare} function.

The default of \cd{compare(x, y) : number} is \cd{(x, y) => x - y}, that is ascending order. While, we may pass \cd{(x, y) => y -x} to order by DESC, or we could pass \cd{(x, y) => y^2 - x^2} to order by DESC of each absolute value.

\mb{isSorted(arr, compare): boolean}
Gets a boolean value indicating whether the \cd{arr} is sorted under this \cd{compare} rule.
\mb{quickSort(arr, compare): [ ]}
Gets a new sorted array by quick sort.
\mb{mergeSort(arr, compare): [ ]}
Gets a new sorted array by merge sort.
\mb{mergeSortBU(arr, compare): [ ]}
Gets a new sorted array by bottom-up merge sort.
\mb{heapSort(arr, option): [ ]}
Gets a new sorted array by heap sort, with \cd{option = {order:"ASC"}} OR \cd{option = {order:"DESC"}}.

\section{Graph Algorithm}
We use \cd{T.Graph} in \cd{t.graph.js} to help implementations of some algorithms in \cd{Graph} static object, which has been put in \cd{graph.*.js}.

All the following methods are invoked like \cd{Graph.*}. To create a graph object, use \cd{new T.Graph*()} 
%(see [APIType](APIType#graph)).

\mb{bfs(graph): [ ]}
Gets the vertex visiting array by BFS order.
\mb{dfs(graph): [ ]}
Gets the vertex visiting array by DFS order.
\mb{dijkstra(graph, s: number = 1): [ ]}
Gets the shortest path length of each vertex from initial vertex $s$.
\mb{multiMinimumCut(graph, times: number): number}
Gets the minimum cut number of graph after that times calculations.
\mb{mstPrim(graph, s: number = 1): number}
Gets the total MST cost of weighted graph by Prim algorithm.
\mb{mstKruskal(graph): number}
Gets the total MST cost of weighted graph by Kruskal algorithm.
\mb{mstKruskal(graph, k): number}
Gets the max space of k-clustering by Kruskal algorithm.
\mb{sccKosaraju(graph): [ ]}
Gets the top $10$ large size strong connect component of directed graph, using Kosaraju algorithm.
\mb{topologicalSort(graph): [ ]}
Gets the topological visiting array of this directed graph.
\mb{undirectedConnected(graph): [ ]}
Gets an array indicating connectivity info of undirected graph. this undirected graph is connected iff  \cd{result.length == 1} AND \cd{result[0][1] == Graph.dfs(graph)}.


\end{document}
