\documentclass{article}
\usepackage{hyperref, listings, color}

% shortcut for scotv's github homepage
\newcommand{\scotv}{https://github.com/scotv}

% shortcut for inline code snippet, like `code` in markdown
% \newcommand{\cd}[1]{\colorbox[rgb]{0.86,0.86,0.86}{\lstinline$#1$}}
\newcommand{\cd}[1]{\lstinline$#1$}

% shortcut for section, subsection, subsubsection
% mb stands for member
\newcommand{\mb}[1]{\subsection*{#1}}
\newcommand{\mmb}[1]{\subsubsection*{#1}}

% set style for multiple lines code snippet
\lstset{numbers=left, numberstyle=\tiny
	, stepnumber=2, numbersep=5pt
	, backgroundcolor=\color[rgb]{0.86,0.86,0.86}
	, basicstyle=\footnotesize\ttfamily
	, breaklines=true}

\title{The API Documents for algo-x}
\author{Scott, Liu}
\date{August, 12, 2014}

\begin{document}
\maketitle

This book has just been migrated from previous wiki page of \href{\scotv/algo-js}{algo-js}. There are lots of formats need to be rebuilt. And, lots of namespaces are still in JaveScript, that I am converting right now.

The purpose of this book is listing the API of projects algo-x:

\begin{itemize}
	\item \href{\scotv/algo-js}{algo-js}: Implementation of some algorithms using JavaScript
	\item \href{\scotv/algo-scala}{algo-scala}: Implementation of some algorithms using Scala
	\item \ldots{} 
\end{itemize}

\section{Type Extension}
All types of structures are defined in \cd{window.T}.

We pass some funtion as parameter sometimes, such as \cd{.map(fn), .forEach(fn), .sort(compare)} \ldots The default \cd{fn} for \cd{.map(fn)} is \cd{x => x}, while the default for comparasion is \cd{(x, y) => x - y}.

\subsection{Type Constructors}
\mmb{new T.LinkedList()}
Constructs a new linked list
\mmb{new T.Stack()}
Constructs a new stack
\mmb{new T.Queue()}
Constructs a new queue
\mmb{new T.MaxHeap(): heap}
Gets a new max heap
\mmb{new T.MinHeap(compare): heap}
Gets a new min heap
\mmb{new T.QuickFind(n: number): unionfind}
Gets a union find by QuickFind algorithm
\mmb{new T.WeightedQuickUnion(n: number): unionfind}
Gets a union find by Weighted QuickFind algorithm
\mmb{new T.BinarySearchTree(compare)}
Gets a Binary Search Tree with the specific comparasion rule
\mmb{new T.Graph(n: number, directed: bool = false)}
Gets a unweighted graph with \cd{n} vertex, which is undirected by default
\mmb{new T.GraphW(n: number, directed: bool = false)}
Gets a weighted graph with \cd{n} vertex, which is undirected by default

\subsection{General Members of Linear Collection}
Linear collection like \cd{LinkedList, Stack, Queue} \ldots have the default traversal order. 

\cd{LinkedList} traverses each element in \cd{0 ... n-1} order;

\cd{Stack} traverses each element in \cd{LIFO} order;

\cd{Queue} traverses each element in \cd{FIFO} order;

There are some general members defined in linear collection. 

\mmb{.size(): number}
Gets the size (length) of the linear collecion.
\mmb{.isEmpty(): boolean}
Returns \cd{true} when the linear collection is empty, otherwise \cd{false}.
\mmb{.forEach(x =\textgreater void): void}
Applies a function to each element in default order.
\mmb{.map(x =\textgreater any): [any]}
Gets a new array mapped from \cd{x => any} in default order.
\mmb{toArray(): [ ]}
Gets a new array containing each element of this linear collection in default order.

\subsection{LinkedList}
\mmb{insert(any, i): void}
Inserts an element after ith element, indexing from 0, if \cd{i >= size()}, insert elem at the end of linear collection.
%
%## Members
%<a name="linkedlist"></a>
%### LinkedList
%Call | Meaning
%:----|:-------
%`size(): number` | the size of the list
%`isEmpty(): boolean` | true when list is empty, otherwise false
%`` | 
%`` | insert elem at the end of list
%`reverse(): void` | reverse the list itself
%`remove(any): void` | remove the first node with elem, if no such node, wo do nothing
%`forEach(x => void): void` | apply the function to each element in `0..n-1` order
%`map(x => any): [any]` |  get a new array mapped from `x => any` in `0..n-1` order
%`toArray(): []` | get a new array containing each element of list in `0..n-1` order
%
%[Back to top](#t)
%
%<a name="stack"></a>
%### Stack
%Call | Meaning
%:----|:-------
%`size(): number` | the size of the list
%`isEmpty(): boolean` | true when list is empty, otherwise false
%`push(any): void` | push item into stack
%`peek(): any` | get the last item from stack, error when empty
%`pop(): any` | get the last item and remove it, error when empty
%`forEach(x => void): void` | apply the function to each element in LIFO order
%`map(x => any): [any]` |  get a new array mapped from `x => any` in LIFO order
%`toArray(): []` | get a new array containing each element of stack in LIFO order
%
%[Back to top](#t)
%
%<a name="queue"></a>
%### Queue
%Call | Meaning
%:----|:-------
%`size(): number` | the size of the list
%`isEmpty(): boolean` | true when list is empty, otherwise false
%`enqueue(any): void` | add item into queue
%`peek(): any` | get the first added item from queue, error when empty
%`dequeue(): any` | get the first added item and remove it, error when empty
%`forEach(x => void): void` | apply the function to each element in FIFO order
%`map(x => any): [any]` |  get a new array mapped from `x => any` in FIFO order
%`toArray(): []` | get a new array containing each element of queue in FIFO order
%
%[Back to top](#t)
%
%<a name="heap"></a>
%### Heap
%Usage of `Sorting.MinHeap` is same as `Sorting.MaxHeap`, except the constructor. Notice, there is a pseudo element at `heap[0]` which we might not use.
%
%The reason we do not implement public `forEach`, `map` or `toArray` is that any of these function will destroy the heap.
%
%Call | Meaning
%:----|:-------
%`isEmpty(): boolean` | returns true if there is no real element in heap
%`size(): number` | gets the number of elements in heap
%`push(any): number` | inserts a element into the heap, and gets length of the new heap
%`pop(): any` | gets the max / min element from heap, and remove it, keeping heap sorted
%
%[Back to top](#t)
%
%<a name="minheap"></a>
%### MinHeap
%Call | Meaning
%:----|:-------
%`update(` <br></a> `key: x=>boolean,` <br></a> `when: x=>boolean,` <br></a> `how: x=>void): boolean` | find the elem holding `key(x)`, if `when(x)`, then update elem by `how(x)`, return false iff no elem hoding `key(x)`
%
%[Back to top](#t)
%
%<a name="quickfind"></a>
%### QuickFind
%When we construct a quick find, we pass `n` as the capacity of the quick find. We index the elements in quick find from `0` to `n-1` (inclusive).
%
%Call | Meaning
%:----|:-------
%`connected(p, q): boolean` | returns true iff `p` is connected with `q`
%`count(): number` | gets the number of components which are not connected with each other
%`union(p, q): number` | unions `p` with `q`, and returns the count after union
%
%Different algorithms of quick find bring us different growth of the cost, see below:
%
%Type | `connected(p, q)` | `union(p, q)`
%:----|:-----------------:|:-------------
%`T.QuickFind(N)` | `O(1)` | `O(N)`
%`T.WeightedQuickFind(N)` | `O(lg N)` | `O(lg N)`
%
%[Back to top](#t)
%
%<a name="binarysearchtree"></a>
%### BinarySearchTree
%We can indicate the comparasion rule by constructor parameter which will be used for greater left instead of smaller left by default.
%
%For `search`, `insert`, `delete` operations which travel the tree, we provide you two versions, one is iterative operation, the other is recursive one whose function name is initialized with a `r`. Always choose iterative operations as a recommandation.
%
%And, here is definition for some properties of tree, according to Wikipedia:
%
%> The height of a node is the length of the longest downward path to a leaf from that node. 
%> 
%> The depth of a node is the length of the path to its root (i.e., its root path). The root node has depth zero, leaf nodes have height zero.
%> 
%> The height of the root is the height of the tree. An empty tree (tree with no nodes, if such are allowed) has depth and height `−1`.
%
%We have defined different traversal way in `T.TRAVERSAL.*`. The default traversal way is `T.TRAVERSAL.IN_ORDER`.
%
%Recursive Call | Iterative Call | Meaning
%:--------------|:---------------|:-------
%`rSearch(elem): node` | `search(elem): node` | searches elem in this tree, returns node which contains elem, or null if not exsits
%`rInsert(elem): void` | `insert(elem): void` | inserts elem under `BST` order, no duplication
%`rForEach(` <br></a> `TRAVERSAL,` <br></a> `fn): void` | `forEach(` <br></a> `TRAVERSAL,` <br></a> `fn): void` | travel this `BST` tree with specific `T.TRAVERSAL` order
%`rMap(` <br></a> `TRAVERSAL,` <br></a> `fn: x => any): [any]` | `map(` <br></a> `TRAVERSAL,` <br></a> `fn): [any]` | maps each element of this `BST` tree into an array with specific `T.TRAVERSAL` order
%
%[Back to top](#t)
%
%<a name="graph"></a>
%### Graph
%Representation of unweighted graph, and it is undirected graph by default.
%
%To build a graph, we must pass `n` as number of vertex. `n`, the number of vertex, is readonly after built graph.
%
%However, when we call `v()` or `e()` as below, the number we get is the valid number of vertex or edges, which has not been visited or marked.
%
%Call | Meaning
%:----|:-------
%`n: number` | gets the actual number of vertex, whether it is visited, marked or not
%`v(): number`| gets the number of vertex, which has not been visited
%`e(): number`| gets the number of edges, which has not been marked invalid, and the sourcing vertex has not been visited
%`clone(): Graph` | gets a new cloned graph from this graph itself
%`toString(` <br></a> `verbose: bool = false)` | gets the information of this graph, containing only number of v and e by default
%
%[Back to top](#t)
%
%<a name="graphw"></a>
%### GraphW
%The members in weighted graph is same as the graph, except some differences between the adjancency list representation:
%
%Type | Element in Adjancency List
%:----|:-------
%`T.Graph` | `[v, [u1, u2, u3, ...]]`
%`T.GraphW` | `[v, [(u1, w1), (u2, w2), (u3, w3), ...]]`
%
%[Back to top](#t)

\section{Array}
The basic extensions for Array.
\subsection{Static}
\begin{tabular}{ll}

Call & Meaning\\
`Array.zip(arr1, arr2): [ ]` & returns an new array, each item in which is `[x, y]`, where `x` from `arr1` and `y` from `arr2`\\
`Array.swap(arr, i, j): void` & swaps the elements at index i and j

\end{tabular}

\subsection{Members}

\begin{tabular}{ll}
Call & Meaning\\
`clone(): []` & gets a new array cloned from itself\\
`zip(that): []` & gets Array.zip(this, that)
\end{tabular}

\section{Math Extension}
The basic extensions for Math.

\mb{Math.mod(i, n): number}
Returns a positive number $x$, where $i = k * n + x, (x > 0)$.
\mb{Math.range(start = 0, end, step = 1): [ ]}
Generates an array of range from \cd{start} (inclusive) to \cd{end} (exclusive), with that \cd{step}.
\mb{Math.randomInteger(a = 0, b): number}
Gets a random integer $x$, where $ a \leq x \leq b $.
 
\subsection{Statistics}
The basic statistics extension that is located in \cd{Math.Stats.*}

\mmb{max(arr): number}
Gets the max number of an array.
\mmb{min(arr): number}
Gets the min number of an array.
\mmb{sum(arr): number}
Gets the sum of an array.
\mmb{mean(arr): number}
Gets the average of an array.
\mmb{var(arr): number}
Gets sample variance of an array.
\mmb{stddev(arr): number}
Gets the standard deviation of an array.
\mmb{linearLeastSquare(X,  Y, fn=(x=\textgreater x) ): [number, number]}
Gets the \cd{[a, b]}, where $fn(Y) = a \cdot fn(X) + b$. passes \cd{Math.log} to get linear least square for $Y = c \cdot X^a$.


\subsection{Math.Point}
The point for the vector.

\mmb{new Math.Point(arr)}
Builds new point by coordinates array.

\subsection{Math.Vector}
The vector in $n$ dimension.

\mmb{new Math.Vector(arr)}
Builds new vector by coordinates array.
\mmb{new Math.Vector(arr1, arr2)}
Builds new vector by coordinates array of two point.
\mmb{new Math.Vector(point1, point2)}
Builds new vector by two Math.Point.

\mmb{Math.Vector.norm(vec): number}
Gets the length of the vec.
\mmb{Math.Vector.dot(v1, v2): number}
Calculates the \cd{dot} result of two vectors.
\mmb{Math.Vector.cos(v1, v2): number}
Calculates the \cd{cos} result of two vectors.

\mmb{norm(): number}
Gets the length of the vec itself.
\mmb{dot(that): number}
Calculates the \cd{dot} result of this and that.
\mmb{cos(that): number}
Calculates the \cd{cos} result of this and that.
\section{Linear Algorithm}
We put some algortithms on linear collection into \cd{List.*}. To use linear collection type, please go to \cd{T.Stack}, \cd{T.LinkdedList} \ldots{}

A overview of time complexity laies below:


% \begin{center}
\begin{tabular}{l | c}
\hline
Call & Time Complexity \\
\hline
\cd{validPopStackSeries} & $O(n)$ \\
\cd{medianMaintenence} & $O(n \ln n)$ \\
\hline
\end{tabular}
% \end{center}

\mb{validPopStackSeries(push: [ ], pop: [ ]): bool}
Returns true iff we can get the \cd{pop} series from the \cd{push} series, using a stack.
\mb{medianMaintenence(arr): [ ]}
For each time $t$ we visit \cd{arr} from $0$ to $n-1$, we maintenence the median of all $[0, t]$ elements from input, and push this median into output array.

\section{Sorting Algorithm}
In our sorting, we pass a function as parameters, named \cd{compare(x, y) : number}. We sort the array by comparing each two value with this \cd{compare} function.

The default of \cd{compare(x, y) : number} is \cd{(x, y) => x - y}, that is ascending order. While, we may pass \cd{(x, y) => y -x} to order by DESC, or we could pass \cd{(x, y) => y^2 - x^2} to order by DESC of each absolute value.

\mb{isSorted(arr, compare): boolean}
Gets a boolean value indicating whether the \cd{arr} is sorted under this \cd{compare} rule.
\mb{quickSort(arr, compare): [ ]}
Gets a new sorted array by quick sort.
\mb{mergeSort(arr, compare): [ ]}
Gets a new sorted array by merge sort.
\mb{mergeSortBU(arr, compare): [ ]}
Gets a new sorted array by bottom-up merge sort.
\mb{heapSort(arr, option): [ ]}
Gets a new sorted array by heap sort, with \cd{option = {order:"ASC"}} OR \cd{option = {order:"DESC"}}.


\end{document}
